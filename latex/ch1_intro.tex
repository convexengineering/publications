\chapter{Introduction}

Modern engineering design, and particularly aerospace design, has come to rely
heavily on optimization. Time and time again, the importance of fast and
reliable MDO tools has been stressed in the literature. However most MDO tools
have the downfalls of being slow due to the multimodal nature of many
engineering design problems.
	
In the \gls{CEG}, we seek to improve the engineering design process by
leveraging the mathematical guarantees and speed of convex optimization. We develop
open-source, object-oriented software to help build \gls{GP}-compatible models and
interface with solvers. In much of our previous work
(\cite{gp_ac_design},\cite{SP_ac_design}, \cite{sp_engine}), we have demonstrated that
\gls{GP} is useful for optimization, but have yet to formalize why it
facilitates design.
 
The mathematical restrictions on the form of constraints remains the biggest
barrier in the implementation of convex programs in design. Firstly, this thesis
will aim to show that the form of the GP actually facilitates the design process
and engineering understanding, rather than impeding it, through the disciplined use
of inequalities to express constraints. Hence it will aim to
pass on some of the expertise we have developed in the \gls{CEG} building
\gls{GP}-compatible models from general non-linear physical models. 

Secondly, it will aim to showcase the extensibility of \gls{GP} that is superior 
to other optimization formulations. The formulation of the \gls{GP} as a 'bag of 
constraints' instead of a hierarchial set of relations confers advantages 
when trying to expand the fidelity and scope of models, especially in the 
conceptual design stage. 

Finally, this thesis will discuss the features specific to GPkit in facilitating
an engineering design process that is streamlined and collaborative, and is
compatible with modern engineering design methodologies. The modularity of the 
models, as well as tools within GPkit to test tightness and boundedness
allows for a component-based design approach that ensures that
requirements both at the sub-system and complete-system levels are satisfied. 
Furthermore, GPkit allows feedback between design engineers and models
which allows targeted efforts by engineers to collaboratively improve models.

\section{Defining Design versus Optimization} \label{s:DesVsOpt}

It is difficult to find rigorous definitions of design and optimization that
try to identify the similarities and differences between the terms. For us to be
able to understand why \gls{GP}s facilitate design, we have to determine what 
features of optimization create barriers to entry for its use in design. 

\subsection{What is design?}

In the context of this paper, I will define design as the following:
To design is to conceive the form and function of something.
In the engineering sense, we think about the form as the configuration or
the parametrization. On the other hand, the function is the actual purpose of the things
being designed. It is oftentimes the aspect of the design that we can
quantify (i.e. the performance), and has some physics that can be modeled.

An important aspect of design is that it is a process that explores a space.
We can think about the feasible set of a design as all of the designs
that satisfy the functional requirements. But without a rigorous way of comparing
the performances of a design, the classical
definition of design implies a class of feasibility problems in a space defined by
the forms that a designer can imagine, given a set
of functional requirements. 

\subsection{What is optimization?}

Optimization has a rigorous mathematical definition: To optimize is to select an 
element in a set of feasible solutions with the lowest desired objective
function value. 

It is also a process, which is sensitive to the elements contained within the set
(related to the form), and the choice of objective function (related to the 
function). In many ways, optimization is a natural extension of design, because it allows
for an explicit mathematical representation of the form and function.

\subsection{The fundamental differences}

To many engineers, design and optimization are one and the same. Both design and
optimization explore feasible and infeasible sets, but differ in two fundamental ways.

Firstly, design is based on feasibility, whereas optimization is based on the
mathematical guarantees of optimality.

Because design does not require any specific mathematical structure, it can
be performed in non-restrictive mathematical forms. Feasibility problems are

Design can be done in non-restrictive mathematical forms. Optimization is
done in specific mathematical forms that take advantage of structure.



\section{Unifying design and optimization using gls{GP}}

\gls{GP} has developed "in response to a need to solve problems in the actual 
world".~\cite{duffingp} Quote from Duffin, 1967, main inventor of GP.

(Each section in this part of the intro has a corresponding section in the body.
(Each will feature an example problem that I will run through as a
(demonstration.)

There are three primary reasons why...

\begin{enumerate}

    \item \textit{Inequalities help engineering understanding.}

The mathematical constraints of \gls{GP} force designers to have a proper grasp
of the fundamental tradeoffs and pressures in a design. Inequalities make feasible
sets explicit.

ADD A DIAGRAM OF FEASIBLE SETS SimPleAC HERE.

Traditionally, physical relations are expressed as equalities. But there is an
almost-seamless transition from fundamental physics to GP-compatible
constraints. 

And even if the direction of pressure on a constraint is not clear,
we can use signomial equalities to explicitly enforce constraints.

    \item \textit{Models are extensible.}
    
Models can be made arbitrarily complex. The 'bag of constraints' form of the GP
means that there is no need to reformulate the optimization scheme as more
constraints are added. This makes incremental modeling improvements straight-forward.
The traditional engineering design process is split into conceptual, preliminary
and critical design segments. GP modeling facilitates this process by allowing
ever-increasing levels of complexity. Gradient based optimization methods for
multimodal systems often involves
converge loops, which have to be reengineered when new constraints are
introduced.

A big advantage of convexity is that we can effectively use sensitivity information
to determine which parts of the
model yield the greatest returns to improved modeling, so engineers can target
their efforts.

    \item \textit{Models are flexible and modular, and compatible with modern
    engineering design.}

The design tools available in GPkit make it easy to
implement component-based modeling, and build models that are shared between
different design problems.

This is helped by the fact that underconstrained \gls{GP}s can solve reliably,
and the tightness of constraints can be monitored.

\end{enumerate}



