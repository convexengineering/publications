\chapter{Introduction}
\label{ch1_intro}

Modern engineering design, and particularly aerospace design, has come to rely
heavily on optimization. Time and time again, the importance of fast and
reliable \gls{MDO} tools has been stressed in the literature. However most \gls{MDO} tools
have the downfall of being slow due to the multimodal nature of many
engineering design problems. Furthermore, these tools act like black boxes since
they provide point solutions without additional information about the design problem
such as sensitivities, which can provide important engineering insight.
	
In the \gls{CEG}, we seek to improve the engineering design process by
leveraging the mathematical guarantees and speed of convex optimization. We develop
open-source, object-oriented software to help build \gls{GP}-compatible models and
interface with solvers. In much of our previous work
(\cite{gp_ac_design},\cite{SP_ac_design}, \cite{sp_engine}), we have demonstrated that
\gls{GP} is useful for optimization, but have yet to formalize why it
facilitates design.
 
The mathematical restrictions on the form of constraints remains the biggest
barrier in using convex programs for design. Firstly, I
aim to show that the form of the GP actually facilitates the design process
and engineering understanding, rather than impeding it, through the disciplined use
of inequalities to express constraints. Hence I will
pass on some of the expertise we have developed in the \gls{CEG} building
\gls{GP}-compatible models from general non-linear physical models. 

Secondly, I aim to showcase the extensibility of \gls{GP}.
The formulation of the \gls{GP} as a 'bag of
constraints' instead of a hierarchial set of relations confers advantages 
when trying to expand the fidelity and scope of models, especially in the 
conceptual design stage. Furthermore, the solution to the dual of the \gls{GP}
provides optimal sensitivities which allow targeted efforts by engineers to
collaboratively improve models.

Finally, this thesis will discuss the features specific to GPkit in facilitating
an engineering design process that is streamlined and collaborative, and is
compatible with modern engineering design methodologies. The modularity of the 
models, as well as tools within GPkit to test tightness and boundedness
allows for a design approach that ensures that
requirements both at the sub-system and complete-system levels are satisfied.

\section{Defining design versus optimization} \label{s:DesVsOpt}

It is difficult to find definitions of design and optimization that
try to identify the similarities and differences between the terms. For us to be
able to understand why \gls{GP}s facilitate design, we have to determine what 
features of optimization create barriers to entry for its use in design. 

\subsection{What is design?}

In the context of this thesis, I will define design as the following:
To design is to conceive the form and function of something.
In the engineering sense, we can think about the form as the configuration or
the parametrization. The form usually defines n, the number of degrees of freedom
of the system, which can have a direct effect on the size of the feasibility
space, as well as the complexity of the problem.
On the other hand, the function is the actual purpose of the things
being designed. It is oftentimes the aspect of the design that we can
quantify (i.e. the performance), and has some physics that can be modeled.

An important aspect of design is that it is a process that explores an
n-dimensional feasible space of possible solutions.
We can think about the feasible set of a design as all of the designs
that satisfy the functional requirements. But without a clear method of comparing
the relative performances of designs, the classical
definition of design implies a class of feasibility problems satisfying a set of
constraints that act on the designer's parametrization of the problem.

\subsection{What is optimization?}

In this thesis, I give optimization the following definition: To optimize is to select an
element in a set of feasible solutions with the lowest desired objective
function value. It is also a process, which is sensitive to the elements contained within the set
(related to the form), and the choice of objective function (related to the 
function).

\subsection{The fundamental differences}

In many ways, optimization is a natural extension of design, because it allows
for an explicit mathematical representation of the form and function. Both design and
optimization explore feasible and infeasible sets, but differ in a fundamental way.
Design is based on feasibility, whereas optimization is based on optimality.
This observations seems banal, but gives insight as to why there is a barrier
to entry for optimization methods (and especially so for more restrictive forms).
The distinction allows design to be performed
in non-restrictive mathematical forms, since non-linear feasibility problems are
much easier to solve than non-linear optimization problems. Optimization is
done in specific mathematical forms; since most problems of interest are complex
computational time is a limited resource. These forms can prove
an impediment for designers unfamiliar with optimization to use it.

\section{Unifying design and optimization using GP}

\gls{GP} has developed "in response to a need to solve problems in the actual 
world"~\cite{duffingp}. \gls{GP}s and other convex optimization methods have been
in development since the 1960's, but have come into the limelight thanks to the development of
polynomial-time algorithms for convex programming~\cite{interior_point} and
improvements in computing. I do not make
the claim that the form of \gls{GP} makes it applicable to every design
problem. Instead, I argue that, for certain kinds of problems, \gls{GP} and convex
optimization naturally integrate into the conceptual design process
for three primary reasons.

\begin{enumerate}

    \item \textit{Inequalities help engineering understanding.}

The mathematical constraints of \gls{GP} force designers to have a proper grasp
of the fundamental tradeoffs and pressures in a design.
    Traditionally, physical relations are expressed as equalities. But there is an
almost-seamless transition from fundamental physics to GP-compatible
inequalities for certain kinds of problems, and the \gls{GP}-compatible
    form makes feasibility sets explicit. Furthermore, the difference-of-convex
    extension of \gls{GP} called \gls{SP} gives us the flexibility to model
    non-log-convex functions as well. Within a \gls{SP},
    we can use signomial equalities to explicitly enforce the tightness of a
    constraint when the direction of pressure is not clear.

    \item \textit{Models are extensible and modular.}
    
Models in \gls{GP} can be made arbitrarily complex. The 'bag of constraints' form of the \gls{GP}
means that there is no need to reformulate the optimization scheme as more
constraints are added. This makes incremental modeling improvements straight-forward.
The traditional engineering design process is split into conceptual, preliminary
and critical design segments. GP modeling facilitates this process by allowing
ever-increasing levels of complexity.

Gradient-based optimization methods for
multimodal, multicomponent systems often involve
converge loops, as shown in Figure~\ref{f:optflow}, which have to be reengineered
    when new constraints are introduced. Furthermore, the designer often has to tune
    the module for generating new guesses from gradient information, which is unreliable
    at best. \gls{GP}s (and the \gls{GP} approximations
    of \gls{SP}s) are solved all-at-once, which means that there are no constraint
    convergence loops to worry about or parameters to tune.

\begin{figure*}[!b]
\begin{subfigure}[b]{0.5\linewidth}
    \begin{center}
    \resizebox{0.75\textwidth}{!}{
\begin{tikzpicture}[node distance=1.5cm, align=center]
    \node (start)        [activityStarts]                   {Initialize variables};
    \node (guess)        [process, below of=start]          {Make a feasible initial guess};
    \node (evaluate)     [process, below of=guess]          {Evaluate design};
    \node (gradient)     [process, below of=evaluate]       {Calculate gradients};
    \node (optimal)      [process, below of=gradient]       {Check optimality condition};
    \node (solution)     [activityRuns, below of=optimal]   {Solution};
    \node (genNew)       [process, left of=gradient, xshift=-3cm]        {Generate new guess};

    \draw[->] (start) -- (guess);
    \draw[->] (guess) -- (evaluate);
    \draw[->] (evaluate) -- (gradient);
    \draw[->] (gradient) -- (optimal);
    \draw[->] (optimal)  -- (solution);
    \draw[->] (optimal) -| (genNew);
    \draw[->]  (genNew) |- (evaluate);
\end{tikzpicture}}
    \end{center}
    \caption{Gradient-based optimization}
\end{subfigure}
\begin{subfigure}[b]{0.5\linewidth}
    \begin{center}
    \resizebox{0.75\textwidth}{!}{
\begin{tikzpicture}[node distance=1.5cm, align=center, scale=0.5]
    \node (start)        [activityStarts]               {Initialize variables};
    \node (guess)        [process, left of=start, xshift=-5cm] {Make initial guess};
    \node (convexify)    [process, below of=guess]      {Convexify};
    \node (optimize)     [process, below of=start]  {Optimize convex problem};
    \node (condition)    [process, below of=optimize]   {Check primal/dual condition};
    \node (solution)     [activityRuns, below of=condition]   {Solution};
    \node (sens)         [activityRuns, left of=solution, xshift=-5cm] {Sensitivities};

    \draw[->] (start) -- node {if DC} (guess);
    \draw[->] (start) -- (optimize);
    \draw[->] (guess) -- (convexify);
    \draw[->] (convexify) -- (optimize);
    \draw[->] (optimize) -- (condition);
    \draw[->] (condition) -- node[yshift=-0.75cm,xshift=1cm] {if DC \& reltol $> \epsilon$} (guess);
    \draw[->] (condition) -- (solution);-
    \draw[->] (condition) -| (sens);
    % TODO: learn how to label the 'if DC' section
\end{tikzpicture}}
    \end{center}
    \caption{Convex, and difference-of-convex optimization}
\end{subfigure}
    \caption{The flow diagrams of two methods of optimization. The 'bag of constraints'
    form of the GP facilitates the addition of variables and constraints while extending
    model capabilities, and provides sensitivities to guide modeling efforts.}
    \label{f:optflow}
\end{figure*}

A big advantage of convexity is that we can effectively use sensitivity information
to determine which parts of the
model yield the greatest returns to improved modeling, so engineers can target
their efforts.

    \item \textit{Models are amenable to mission and multi-point design, and compatible with modern
    engineering design.}

The design tools available in GPkit make it easy to
implement mission design, and build models that are shared between
different design problems. Mission design helps engineers gain valuable intuition about
the tradeoffs in the performance of a design, and multi-point design allows designs to
be able to fulfill a variety of missions. Furthermore, certain objective functions can allow
    for the design of both the mission requirements and the configuration simultaneously (eg.
    the design of an aircraft configuration along with its range and payload to optimize
    fuel burn per passenger mile).

\end{enumerate}

In this thesis, I will methodically demonstrate the advantages of \gls{GP} in modeling
and exploring complex engineering trade spaces.



