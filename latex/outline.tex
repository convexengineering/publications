\documentclass{aiaa-pretty}
\usepackage{graphicx}
\usepackage{amssymb}
\usepackage{amsmath}
\usepackage{calc}
\usepackage{supertabular}
\usepackage{subfigmat}
\usepackage{array}
\usepackage{textcomp}
\usepackage{hyperref}
\usepackage[acronym]{glossaries}
\pagestyle{empty}
\makeglossaries

%% Create hyperlinks
\usepackage{hyperref}
\hypersetup{
    colorlinks,
    citecolor=black,
    filecolor=black,
    linkcolor=black,
    urlcolor=black
}
%%
\newacronym{CE}{CE}{Convex Engineering}
\newacronym{CEG}{CEG}{Convex Engineering Group}
\newacronym{GP}{GP}{Geometric Programming}
\newacronym{gp}{GP}{Geometric Program}
\newacronym{SP}{SP}{Signomial Programming}
\newacronym{sp}{SP}{Signomial Program}
\newacronym{mdo}{MDO}{Multidisciplinary Design Optimization}

\newacronym{NLP}{NLP}{Nonlinear Programming}

\begin{document}

\title{Organic integration of conceptual design and optimization using geometric programming}
\author{Berk \"Ozt\"urk}
\maketitle

\section{\bf Introduction}

Modern engineering design, and particularly aerospace design, has come to rely heavily on optimization. Time and time again, the importance of fast and reliable MDO tools has been stressed in the literature. However most MDO tools have the downfalls of being slow due to the multimodal nature of many engineering design problems.  
	
In the \gls{CEG}, we seek to improve the engineering design process by leveraging the mathematical guarantees of convex optimization, and developing open-source, object-oriented software to help build GP-compatible models and interface with solvers. In much of our previous work (\cite{gp_ac_design},\cite{sp_ac_design}, York), we have demonstrated that \gls{GP} is useful for optimization, but have yet to formalize why it facilitates design.
 
The mathematical restrictions on the form of constraints remains the biggest barrier in the implementation of convex programs in design. Firstly, this thesis will aim to show that the form of the GP actually facilitates the design process and engineering understanding, rather than impeding it. Hence it will aim to pass on some of the expertise we have developed in the \gls{CEG} building \gls{GP}-compatible models. 

Secondly, it will aim to showcase the features of GPkit in further facilitating an engineering design process that is streamlined and collaborative, and is compatible with modern engineering design methodologies (flexibility and modularity). GPkit allows feedback between design engineers and models (speed and sensitivities) which allows targeted efforts by engineers to improve models. 

(NED Comment: What about conceptual design? Collaboration?)

\subsection{Defining Design versus Optimization}

Firstly, I aim to go back to the fundamentals of design and optimization and determine why there is a barrier to entry in using convex optimization methods.

(Haven't really figured out what to do with this yet.)

\begin{itemize}
\item What is design? 
\begin{itemize}
\item To conceive the look (the form) and function of something. 
\item Ned's comment: Parametrization and physics. Hard to agree on a representation that we all agree on. 
\item It is a PROCESS
\begin{itemize}
	\item The 'look' is the configuration
	\item The function is oftentimes what we can quantify
\end{itemize}
\item Essentially a class of feasibility problems, given a set of requirements. 
\end{itemize}
\item What is optimization?
\begin{itemize}
\item Has a rigorous mathematical definition: The selection of an element in a set of feasible solutions with the lowest desired objective function value. 
\item It is also a process!
\item It is sensitive to the choice of objective function, and the elements contained within the set (configuration)
\item To many engineers, design and optimization are one and the same. 
\end{itemize}
\item What are the fundamental differences? 
\begin{itemize}
\item Design is human-driven. Optimization is computational. 
\item Design is based on feasibility. Optimization is based on the mathematical guarantees of optimality. 
\item Design can be done in non-restrictive mathematical forms. Optimization is done in specific mathematical forms that take advantage of structure. 
\end{itemize}
\end{itemize}
\subsection{Design and Optimization using \gls{GP}}

\gls{GP} has developed "in response to a need to solve problems in the actual world".~\cite{duffingp} Quote from Duffin, 1967, main inventor of GP. 

(Each section in this part of the intro has a corresponding section in the body. Each will feature an example problem that I will run through as a demonstration.) 

\subsubsection{Inequalities help engineering understanding.}
The mathematical constraints of \gls{GP} force designers to have a proper grasp of the fundamental tradeoffs and pressures in a design.

Inequalities make feasible sets explicit. ADD A DIAGRAM OF FEASIBLE SETS using SimPleAC HERE. 

Traditionally, physical relations are expressed as equalities. But there is an almost-seamless transition from fundamental physics to GP-compatible constraints. ENGINEERS LIKE +VE QUANTITIES. eg. downforce vs. negative lift. 

And even if the tradeoffs are not clear, we can use signomial equalities to enforce constraints. 

\subsubsection{Models are extensible.}
Models can be made arbitrarily complex. The 'bag of constraints' form of the GP means that there is no need to reformulate the optimization scheme as more constraints are added. This makes incremental modeling improvements possible and even preferable. 

The traditional engineering design process is split into conceptual, preliminary and critical design segments. GP modeling facilitates this process by allowing ever-increasing levels of complexity.

Gradient based optimization methods for multimodal systems often involves converge loops, which have to be reengineered when new constraints are introduced. VS. adding constraints to a bag...

We can effectively use sensitivity information to determine which parts of the model yield the greatest returns to improved modeling, so engineers can target their efforts. 

\subsubsection{Models are flexible and modular, and compatible with modern engineering design (similar to extensible?)}
(This is a feature of GPkit, and not GPs in general.)
\gls{GP}s make it easy to implement component-based modeling, and build models that are shared between different design problems. Each component has associated sizing and performance variables. 

This is helped by the fact that underconstrained \gls{GP}s can solve reliably, and the tightness of constraints can be monitored. 

\section{Literature Review}
(On GP modeling and benefits of convex programming.)
Papers to include
\begin{itemize}
\item Duffin, 1967
\item Boyd GP Tutorial
\item Hoburg, 2013: GP aircraft models
\item Kirschen, 2015: SP aircraft models, big increase in complexity and fidelity
\item York, 2016: Engine model, thinking about hierarchy
\end{itemize}
\subsection{Benefits of convex optimization}
To an engineer, in the order of importance:
\begin{itemize}
\item Sensitivities
\item Restrictive forms can be overcome using difference-of-convex programs
\item Speed
\item Global optimality
\end{itemize}

\section{Engineering inequalities and intuition, from equalities}

Example problem: Deriving the simple SP aircraft problem. 

Ideas of tightness, boundedness, and pressure explained. 

Explaining when \gls{SP} constraints are required. 

Giving an example where signomial equalities are required. 

\section{Extensibility of \gls{GP}}

Example problem: Creating a fuselage model for a commercial aircraft, or adding a higher fidelity engine model for the simple SP aircraft. 

\section{Flexibility and modularity of GPkit models}

Example problem: Conventional aircraft tail model extended to pi-tails. 

\subsection{Component breakdown}

\subsection{Functional breakdown}

\subsection{Static vs. performance variables}

\section{Visual debugging of \gls{GP} models}

(Potentially) Expose people to the beauty of automatic CAD generation. 

\section{Conclusion}

\end{document}
