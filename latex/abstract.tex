% $Log: abstract.tex,v $
% Revision 1.1  93/05/14  14:56:25  starflt
% Initial revision
% 
% Revision 1.1  90/05/04  10:41:01  lwvanels
% Initial revision
% 
%
%% The text of your abstract and nothing else (other than comments) goes here.
%% It will be single-spaced and the rest of the text that is supposed to go on
%% the abstract page will be generated by the abstractpage environment.  This
%% file should be \input (not \include 'd) from cover.tex.


Geometric programs (GPs) and other forms of convex optimization have recently experienced
a resurgence due to the advent of polynomial-time solution algorithms and improvements in computing.
Observing the need for fast and stable methods for multidisciplinary
design optimization (MDO),
previous work has shown that geometric programming can be a powerful framework
for MDO by leveraging the mathematical guarantees
and speed of convex optimization. However, there are barriers to
the implementation of optimization in design.
In this work, we formalize how the formulation
of non-linear design problems as GPs facilitates design process.
Using the principles of pressure and boundedness,
we demonstrate the intuitive transformation of core physical and
data-based relations into GP-compatible constraints by systematically formulating an aircraft
design model. We motivate the difference-of-convex GP extension called signomial programs (SPs)
in order to extend the scope and fidelity of the model.
We detail the features specific to GPkit, an object-oriented GP formulation framework, which
facilitate the modern engineering design process.
Using both performance and mission modeling paradigms, we demonstrate the ability to
model and design increasingly complex systems in GP, and extract maximal engineering intuition
using sensitivities and tradespace exploration methods.
Though the methods are applied to an aircraft design problem, they are general to
models with continuous, explicit constraints, and lower the barriers to implementing
optimization in design.

