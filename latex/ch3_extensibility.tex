\chapter{Extensibility of GP}

Traditional engineering design optimization tools (eg. TASOPT) implement
convergence loops that assume structure within a given design problem.
The 'bag of constraints' form of the GP means that constraints can be
added to the problem without
having to restructure the optimization formulation.

\section{Improving fidelity: Adding a simple engine model to the SimPleAC}
\label{s:engine}

The SimPleAC currently has an engine that weighs nothing and magically supplies
unlimited power. This is obviously unphysical, and since one of the variables
with the highest sensitivity is \BSFC, requires refinement.

\subsection{Creating an engine submodel}

Before even thinking about modeling, we would like to leverage the object-oriented 
GPkit models to put the variables describing the engine into a submodel (currently only 
BSFC). We do this by creating a new class called \textbf{Engine} and creating a \textit{setup}
method that returns the constraints within it.

\begin{python}
class Engine(Model):
    def setup(self):
        # Dimensional constants
        BSFC      = Variable("BSFC", 0.6, "1/hr", "brake specific fuel consumption")
        constraints = []
        return constraints    
\end{python}

We allow the SimPleAC model to contain the variables and constraints of the engine
as follows:

\begin{python}
class SimPleAC(Model):
    def setup(self):
        self.engine = Engine()
        self.components = [self.engine]
        # Env. constants
        ...
        return constraints, self.components
\end{python}

This restructuring of the model yields the exact same overall GP formulation 
as the unstructured problem, but gives us the flexibility to develop submodels
collaboratively and in a disciplined manner.

If we think of an engine as an input-output system, we can determine how it
would interact with the SimPleAC system, and create the appropriate variables. 
At the most basic level, the engine provides shaft power, consumes fuel, 
and has weight. The model is missing both the shaft power and weight description
of the engine. If we abstract away the propeller (the relation between shaft
power and thrust power) for the moment, we can perhaps find a way to relate
power to weight

\subsection{Data-based modeling: engine power vs. weight}
\label{s:datafit}

We can imagine that, for a specific kind of engine, there is a relation between the 
maximum shaft power available and the mass of the engine, somewhat related to the 
cube-square law, which describes the relation between the surface area and volume 
of objects. And let's say that our knowledge of the internal workings of engines
is limited, but we have some knowledge of the technology available in the market
and have data to support it. Using GPfit, we will try to fit the data to find
\gls{GP} compatible relations between engine weight and maximum power. This section
will try to highlight the best practices when making data-based models. 

To be able to fit the engine power vs. weight data, we take several important steps.
\begin{itemize}
    \item \textbf{Comb the data.} Since we are essentially projecting
    data with potentially high standard deviation into a single line,
    it is important to fit the ranges of data we care about.
    \item \textbf{Normalize the data.} It is very important to normalize the data
	by some known quantity, since we would not want our fit to be dependent on the
	units that we used while performing it. This also helps the fit integrate
    seamlessly into GPkit, since dimensional fits would require units manipulation
    to avoid errors. The data can be normalized by any
    reference quantities (in this case using the maximum power and weight values
    from the data set).
    \item \textbf{Choose the type of fit.} In~\cite{gpfitpaper}, \textit{softmax-affine}
    (SMA) and \textit{implicit softmax-affine} (ISMA)
    functions are proposed as convex approximations
    to data. Depending on the behavior of the data, one or the other
    may be appropriate. For engineering relations that are expected to be smooth, SMA
    functions are often good approximations. However, if kinks are expected in the
    functions, ISMA functions can locally adjust the softness of the fit to
    reduce the error of the fit.
    \item \textbf{Choose the number of posynomial terms in the fit.} The number of
    terms will likely depend on the root-mean square (RMS) error of the fit, and
    the kind of pressure on the variable. RMS error can be reduced by including
    more posynomial terms, but only if the variable of interest has downward
    pressure on it from the objective function (since it is on the greater side
    of the inequality).
\end{itemize}


The relation we obtain for the one-term approximation is as follows:
\begin{equation}
	(\frac{W_{eng}}{W_{eng_{max}}})^{0.100} = 0.988 (\frac{P_{shaft}}{P_{shaft_{max}}})^{0.117}
\end{equation}

Note that the root mean square error of this fit is 0.414, but this has to do with 
the level of variation in the data. Since engine weight will have downward pressure
on it from the objective, we can easily use a two-term posynomial approximation to 
improve its error. 

\begin{equation}
	(\frac{W_{eng}}{W_{eng_{max}}})^{0.801} \geq 0.0330 (\frac{P_{shaft}}{P_{shaft_{max}}})^{0.167} 
												+1.59 (\frac{P_{shaft}}{P_{shaft_{max}}})^{1.36}
\end{equation}

This relation has an r.m.s. error of 0.346, which is a significant improvement. 

It turns out that with a softmax affine approximation, more terms do not improve the r.m.s.
error of the fit on the given data. As such, we will proceed with the 2-term posynomial fit. 

\subsection{Converting all subsystems into submodels}
\label{s:submodels}

Within this framework, we can modularize also add a separate wing and fuselage module to
SimPleAC as well, with very little additional work. This creates the variable and constraint
hierarchy as presented in Figure~\ref{forest:submodels}, which define all of the constraints
required for SimPleAC to fly one flight segment.

\begin{figure}[!h]\centering\small\sffamily
\begin{forest}
    [\textbf{Aircraft}
        [\textbf{Wing}]
        [\textbf{Fuselage}]
        [\textbf{Engine}]
    ]
\end{forest}
\caption{Variable and constraint hierarchy of the single mission segment SimPleAC model}
\label{forest:submodels}
\end{figure}

\section{Designing missions: Converting the SimPleAC to performance modeling form}
\label{s:mission}

The SimPleAC that I have defined so far works well to demonstrate the
capabilities of \gls{SP} in helping explore tradeoffs in engineering design.
However, often in the design process, we will want to test the performance of a
design in different conditions. This requires the vectorization of
constraints that relate to the performance of the design.

To design the mission for SimPleAC, we have to have a better understanding
of the environment in which the aircraft operates. So far, we have assumed that
the aircraft flies at a constant altitude (sea level) for a single mission segment,
and is subject to the same air density and viscosity. What we'd like to do instead
is to have a single aircraft optimize both its static sizing variables (having
to do with the airframe), and its flight performance. This requires a major
augmentation of the model tree defined in Figure~\ref{forest:submodels}.

\begin{figure}[!h]\centering\small\sffamily
\begin{forest}
        [\textit{\textbf{Mission}}
            [\textit{Atmosphere}]
            [\textit{\textbf{\shortstack{Aircraft\\Perf.}}}
                [\textbf{Aircraft}
                    [\textbf{Wing}]
                    [\textbf{Fuselage}]
                    [\textbf{Engine}]
                ]
                [\textit{\shortstack{Wing\\Perf}}]
                [\textit{\shortstack{Engine\\Perf.}}]
            ]
        ]
\end{forest}
   \caption{Variable and constraint hierarchy of the presented aircraft model. Models that include sizing variables are
bolded while models that include performance variables are italicized.
There are models that contain both kinds of variables.}
\label{f:componenttree}
\end{figure}

Within the model hierarchy we have identified, we have to determine which variables
belong in which level of the tree. Table

\begin{center}
\captionof{table}{Variables of SimPleAC in performance modeling, detailed in the
variable and constraint hierarchy.}
{\footnotesize
\begin{longtable}{lcl}
\toprule
Free Variables & Units & Description \\ \midrule
\multicolumn{3}{l}{\textbf{Mission}} \\
$W_{f_{m}}$ & $~\mathrm{N}$ & Total mission fuel \\
$t_m$ & $~\mathrm{hr}$ & Total mission time \\
$R_s$ & $~\mathrm{km}$ & Range flown in segment \\
$W_{avg}$ & $~\mathrm{N}$ & Segment average weight \\
$W_{end}$ & $~\mathrm{N}$ & Weight at the end of flight segment \\
$W_{f_s}$ & $~\mathrm{N}$ & Segment fuel burn \\
$W_{start}$ & $~\mathrm{N}$ & Weight at the beginning of flight segment \\
$\frac{\Delta h}{dt}$ & $~\mathrm{\tfrac{m}{hr}}$ & Climb rate \\
$h$ & $~\mathrm{m}$ & Flight altitude \\
$t_s$ & $~\mathrm{hr}$ & Time spent in flight segment \\
\hline
\multicolumn{3}{l}{\textbf{Mission/Atmosphere}} \\
$\mu$ & $~\mathrm{\tfrac{kg}{\left(m\cdot s\right)}}$ & dynamic viscosity \\
$\rho$ & $~\mathrm{\tfrac{kg}{m^{3}}}$ & density of air \\
$h$ & $~\mathrm{m}$ & altitude \\
\hline
\multicolumn{3}{l}{\textbf{Mission/SimPleAC}} \\
$V_f$ & $~\mathrm{m^{3}}$ & maximum fuel volume \\
$V_{f_{avail}}$ & $~\mathrm{m^{3}}$ & fuel volume available \\
$W$ & $~\mathrm{N}$ & maximum takeoff weight \\
$W_f$ & $~\mathrm{N}$ & maximum fuel weight \\
\hline
\multicolumn{3}{l}{\textbf{Mission/SimPleAC/Engine}} \\
$P_{shaft_{max}}$ & $~\mathrm{kW}$ & MSL maximum shaft power \\
$W_e$ & $~\mathrm{N}$ & engine weight \\
\hline
\multicolumn{3}{l}{\textbf{Mission/SimPleAC/Fuselage}} \\
$(CDA0)$ & $~\mathrm{m^{2}}$ & fuselage drag area \\
$C_{D_{fuse}}$ & $$ & fuselage drag coefficient \\
$V_{f_{fuse}}$ & $~\mathrm{m^{3}}$ & fuel volume in the fuselage \\
\hline
\multicolumn{3}{l}{\textbf{Mission/SimPleAC/Wing}} \\
$A$ & $$ & aspect ratio \\
$S$ & $~\mathrm{m^{2}}$ & total wing area \\
$V_{f_{wing}}$ & $~\mathrm{m^{3}}$ & fuel volume in the wing \\
$W_w$ & $~\mathrm{N}$ & wing weight \\
$W_{w_{strc}}$ & $~\mathrm{N}$ & wing structural weight \\
$W_{w_{surf}}$ & $~\mathrm{N}$ & wing skin weight \\
\hline
\multicolumn{3}{l}{\textbf{Mission/SimPleACP}} \\
$C_D$ & $$ & drag coefficient \\
$D$ & $~\mathrm{N}$ & total drag force \\
$L/D$ & $$ & lift-to-drag ratio \\
$Re$ & $$ & Reynolds number \\
$V$ & $~\mathrm{\tfrac{m}{s}}$ & cruising speed \\
\hline
\multicolumn{3}{l}{\textbf{Mission/SimPleACP/EngineP}} \\
$P_{shaft}$ & $~\mathrm{kW}$ & shaft power \\
$T$ & $~\mathrm{N}$ & propeller thrust \\
\hline
\multicolumn{3}{l}{\textbf{Mission/SimPleACP/WingP}} \\
$C_L$ & $$ & wing lift coefficient \\
$C_f$ & $$ & skin friction coefficient \\
$C_{D_{ind}}$ & $$ & wing induced drag \\
$C_{D_{wpar}}$ & $$ & wing profile drag \\
\bottomrule
\end{longtable}}

{\footnotesize
\begin{longtable}{lcl}
\toprule
Constants & Units & Description \\ \midrule
\multicolumn{3}{l}{\textbf{Mission}} \\
$Cost Index$ & $~\mathrm{\tfrac{1}{hr}}$ & hourly cost index \\
$Range$ & $~\mathrm{km}$ & aircraft range \\
$V_{min}$ & $~\mathrm{\tfrac{m}{s}}$ & takeoff speed \\
$W_p$ & $~\mathrm{N}$ & payload weight \\
\hline
\multicolumn{3}{l}{\textbf{Mission/Atmosphere}} \\
$P_{MSL}$ & $~\mathrm{Pa}$ & pressure at MSL \\
$T_{MSL}$ & $~\mathrm{K}$ & temperature at MSL \\
$\mu_{MSL}$ & $~\mathrm{\tfrac{kg}{\left(m\cdot s\right)}}$ & dynamic viscosity at MSL \\
$\nu_{MSL}$ & $~\mathrm{\tfrac{m^{2}}{s}}$ & kinematic viscosity at MSL \\
$\rho_{MSL}$ & $~\mathrm{\tfrac{kg}{m^{3}}}$ & density of air at MSL \\
$a_{MSL}$ & $~\mathrm{\tfrac{m}{s}}$ & Speed of sound at MSL \\
$h_{top}$ & $~\mathrm{m}$ & highest altitude valid \\
\hline
\multicolumn{3}{l}{\textbf{Mission/SimPleAC}} \\
$\rho_f$ & $~\mathrm{\tfrac{kg}{m^{3}}}$ & density of fuel \\
$g$ & $~\mathrm{\tfrac{m}{s^{2}}}$ & gravitational acceleration \\
\hline
\multicolumn{3}{l}{\textbf{Mission/SimPleAC/Engine}} \\
$P_{shaft_{ref}}$ & $~\mathrm{kW}$ & reference MSL maximum shaft power \\
$W_{e_{ref}}$ & $~\mathrm{N}$ & reference engine weight \\
$\eta_{prop}$ & $$ & propeller efficiency \\
\hline
\multicolumn{3}{l}{\textbf{Mission/SimPleAC/Wing}} \\
$(\frac{S}{S_{wet}})$ & $$ & wetted area ratio \\
$C_{L,max}$ & $$ & max CL with flaps down \\
$N_{ult}$ & $$ & ultimate load factor \\
$W_{w_{coeff1}}$ & $~\mathrm{\tfrac{1}{m}}$ & wing weight coefficent 1 \\
$W_{w_{coeff2}}$ & $~\mathrm{Pa}$ & wing weight coefficent 2 \\
$\tau$ & $$ & airfoil thickness to chord ratio \\
$e$ & $$ & Oswald efficiency factor \\
$k$ & $$ & form factor \\
\hline
\multicolumn{3}{l}{\textbf{Mission/SimPleACP/EngineP}} \\
$BSFC$ & $~\mathrm{\tfrac{g}{\left(hr\cdot kW\right)}}$ & thrust specific fuel consumption \\
\bottomrule
\end{longtable}}
{\footnotesize
\begin{longtable}{lcl}
\toprule
Sensitivities & Units & Description \\ \midrule
\multicolumn{3}{l}{\textbf{Mission}} \\
$Range$ & $~\mathrm{km}$ & aircraft range \\
$V_{min}$ & $~\mathrm{\tfrac{m}{s}}$ & takeoff speed \\
$Cost Index$ & $~\mathrm{\tfrac{1}{hr}}$ & hourly cost index \\
$W_p$ & $~\mathrm{N}$ & payload weight \\
\hline
\multicolumn{3}{l}{\textbf{Mission/Atmosphere}} \\
$\rho_{MSL}$ & $~\mathrm{\tfrac{kg}{m^{3}}}$ & density of air at MSL \\
$\mu_{MSL}$ & $~\mathrm{\tfrac{kg}{\left(m\cdot s\right)}}$ & dynamic viscosity at MSL \\
$h_{top}$ & $~\mathrm{m}$ & highest altitude valid \\
\hline
\multicolumn{3}{l}{\textbf{Mission/SimPleAC}} \\
$g$ & $~\mathrm{\tfrac{m}{s^{2}}}$ & gravitational acceleration \\
$\rho_f$ & $~\mathrm{\tfrac{kg}{m^{3}}}$ & density of fuel \\
\hline
\multicolumn{3}{l}{\textbf{Mission/SimPleAC/Engine}} \\
$\eta_{prop}$ & $$ & propeller efficiency \\
$P_{shaft_{ref}}$ & $~\mathrm{kW}$ & reference MSL maximum shaft power \\
$W_{e_{ref}}$ & $~\mathrm{N}$ & reference engine weight \\
\hline
\multicolumn{3}{l}{\textbf{Mission/SimPleAC/Wing}} \\
$(\frac{S}{S_{wet}})$ & $$ & wetted area ratio \\
$k$ & $$ & form factor \\
$C_{L,max}$ & $$ & max CL with flaps down \\
$e$ & $$ & Oswald efficiency factor \\
$\tau$ & $$ & airfoil thickness to chord ratio \\
$W_{w_{coeff2}}$ & $~\mathrm{Pa}$ & wing weight coefficent 2 \\
$N_{ult}$ & $$ & ultimate load factor \\
$W_{w_{coeff1}}$ & $~\mathrm{\tfrac{1}{m}}$ & wing weight coefficent 1 \\
\hline
\multicolumn{3}{l}{\textbf{Mission/SimPleACP/EngineP}} \\
$BSFC$ & $~\mathrm{\tfrac{g}{\left(hr\cdot kW\right)}}$ & thrust specific fuel consumption \\
\bottomrule
\end{longtable}}

\end{center}

\section{Multimission design}
