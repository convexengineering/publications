\documentclass{aiaa-pretty}
\usepackage{graphicx}
\usepackage{amssymb}
\usepackage{amsmath}
\usepackage{calc}
\usepackage{supertabular}
\usepackage{subfigmat}
\usepackage{array}
\usepackage{textcomp}
\usepackage{hyperref}
\pagestyle{empty}

\begin{document}

\title{Organic integration of conceptual design and optimization using GPkit}
\section{\bf Introduction}
	What motivates the creation of  fast MDO tool? 
	
\subsection{Defining Design versus Optimization}
\begin{itemize}
\item What is design? 
\begin{itemize}
\item To conceive the look (the form) and function of something. 
\item It is a PROCESS
\begin{itemize}
	\item The 'look' is the configuration
	\item The function is oftentimes what we can quantify
\end{itemize}
\item Essentially a class of feasibility problems 
\begin{itemize}
\item Given set of requirements
\end{itemize}
\end{itemize}
\item What is optimization?
\begin{itemize}
\item Has a rigorous mathematical definition: The selection of an element in a set of feasible solutions with the lowest desired objective function value. 
\item It is also a process!
\item It is sensitive to the choice of objective function, and the elements contained within the set (configuration)
\end{itemize}
\end{itemize}
\subsection{Design and Optimization using Geometric Programming}

\begin{itemize}
	\item Insert Duffin quote here. 
\end{itemize}
\subsubsection{Inequalities help engineering understanding.}
The mathematical constraints of geometric programming force designers to have a proper grasp of the fundamental tradeoffs in a design.

Traditionally, physical relations are expressed as equalities.  

There is an almost-seamless transition from fundamental physics to GP-compatible constraints.

And even if the tradeoffs are not clear, we can use signomial equalities to enforce constraints. 

\subsubsection{Models are extensible.}
Models can be made arbitrarily complex.
\subsubsection{Models are flexible and modular.}
Underconstrained models can solve reliably, and the tightness of constraints can be monitored. 
	
\section{Literature Review}
Papers to include
\begin{itemize}
\item Duffin, 1967
\item Boyd GP Tutorial
\item Hoburg, 2013
\item Kirschen, 2015
\item York, 2016
\end{itemize}
\subsection{Benefits of convex optimization}
To an engineer, in the order of importance:
\begin{itemize}
\item Sensitivities
\item Restrictive forms can be overcome using difference-of-convex programs
\item Speed
\item Global optimality
\end{itemize}

We have demonstrated that geometric programming is useful, but have yet to formalize why it is preferable from an engineering design standpoint. 

\section{Engineering inequalities and intuition, from equalities}

Example problem: Deriving the simple SP aircraft problem. 

Ideas of tightness, boundedness, and pressure explained. 

\section{Extensibility of geometric programming}

Example problem: Creating a fuselage model for a commercial aircraft. 

\section{Modularization}

Example problem: Conventional aircraft tail model extended to pi-tails. 

\end{document}